\section{Introduction}
Problem statement:
Recently in China, 
there are a great deal of controversies about using pictures of others that is unauthorised.
This leads to a concept called image copyright.
Image copyright is used to protect the images owned by poeple or companies from copy or reproduce by others automatically.[https://www.pixsy.com/academy/image-user/using-copyrighted-images/]
However, a large quantity of firms and individuals didn't obey the rules.
This stealing problem has been troubled a large amount of photographers, firms and companies.
Many of them went on a court inorder protect their works by means of law.
But this weren't work so well because althogh there are copyright, 
there are still poeple using illegal images  without permisiion.
According to the information that was published on the website poeple.com.cn,
From<China Online Copyright Protection Annual Report 2018>, 
it pointed out that the report pointed out that copyright infringement cases of picture works have exploded, 
plagiarism and illegal reprinting on public accounts are serious.
In 2018, China made outstanding achievements in the protection of online copyright. 
During the "Swords Net 2018" special action, copyright law supervision departments investigated and dealt with 544 online infringement and piracy cases, 
including 74 criminal cases with a total value of 150 million RMB.

Here is a example recent years in China.
There was a company called ChinaVisual.
On Apr.11th,2019, 
the China Visual Group(CVG) has been alleged to have published with its own watermark on their website, 
this made the cyberspace affairs to suspend its website.
But the problem is, the copyright of this picture belongs to an organization called Event Horizon Telescope. 
This organization has first released the picture to the whole world and made a claim that was not meant for any commercial use.
More over, people found other images of national flags which definitely not belongs to ChinaVisual.
There are also discussions on a chinese social media called weibo about the image copyright of this incident. 
The Central Committee of the Communist Young League warned the ChinaVisual by using rhetorical question--Are those images of national flags yours?
This kind of incidents didn't just happen in China, things like stealing and using images illegally happens all over the world.

Nowadays, people are putting more importance on the image copyright problems and this lead us to think,
are there better and more effective way to protect the images?
The objective of this paper is to conduct a survey regarding to solve the issue of protection of image copyright and explain how they are using the technologies.

\section{Related Works}
There have been many things can be done in the protection of the image copyright. 
I seperate this section into three catagories: 
-Reverse image searching
-Watermarks-(fourier, basic RBG, fourier, frequency domain)
-Image clustering and so on.

\First{Reverse image searching}
There are many cases that poeple don't know their image has been used by other illegally.
Therefore, people need tools to find out how can they know that their works has been stolen.

The best mothod is reverse image searching.
Reverse image searching is a search engine technology that takes an image file as input query and returns results related to the image.[https://whatis.techtarget.com/definition/reverse-image-search][Gaillard, Mathieu, Elöd Egyed-Zsigmond, and Michael Granitzer. "CNN features for Reverse Image Search." Document numerique 21.1 (2018): 63-90.]
To better understand reverse image search, there are two concept:
1) Image Retrieval
Image retrival is a system that normally used in the browsing.
By searching and retrieving images from the large database of images.
Examples like Baidu Image and Google image, users input text queries and will have a list of images which matches the text queries in return.
Those images are assosated with text such as keywords, tags and so on.
2) Content-based Image Retrieval(CBIR)
Reverse image search(RIS) is include in this system.
The aim of this system is to find a image that similar to the input images.
Examples for this system are Google images or Baidu Image and so on,
if someone submits a picture of a cat, in return he can get a list of similar images that similar to the input image but nor exacly the same.
Other mature reverse searching engine consisting Tineye, Noobox and so on.

Using reverse image search can get these imformations:[https://tineye.com/faq#why]
-Find sources of images
-Identify duplicate images(illegally stolen)
-Find higher resolution of images
-Locate web pages of the images

There are a plenty of good algorithms to attain the reverse image searching.
The base of the reverse image seaching is do comparsion among images.
1) Color-based
Most of the comparson algorithm is based on colors.
The method is simply find the red, green, blue value(RGB value) for each images and do comparison.
RGB value is an additive color model in which red, green, and blue light are added together in various ways to reproduce a broad array of colors.
This method first collects the RGB value in each of the images and do comparison within them.

2) Deep learning(CNN)
Activations of units within top layers of convolutional neural networks (CNN features) are known to be good descriptors for image retrieval.[Gaillard, Mathieu, Elöd Egyed-Zsigmond, and Michael Granitzer. "CNN features for Reverse Image Search." Document numerique 21.1 (2018): 63-90.]
There are three steps by using the deep learning(CNN).[https://vitalflux.com/reverse-image-search-using-deep-learning-cnn/]
1. Creat embeddinds in other word numerical feature vector that can reoresent the images.
2. Once the embeddings are created by the autoencoder network(CNN), these embeddings will need be stored into the data base.
3. Creat a nearest neighbor search algorithm that can search from the database that match the input image.

The information of the images will be extracted by many algorithms, 
these informations are mostly presented the by p-dimentional vector of numerical features or a q-bit binary code.
The informations can be devided into low-level feature and high level features, 
high level feature is difficult to extract for computers becasue it's hard to understand.
Low level language which shows the details of the image that will be extracted by algorithms such as corners, edges and so on.
In the last part of the procedure(distance or comparison), if the image representation is p-dimentional vectors,
there are two algorithms: Minkowski distance and cosine similarity. If the image representation is binary codes, 
there are also one algorithm: Hamming distance.

3)Using Pixel tolerance:
[https://support.smartbear.com/testcomplete/docs/testing-with/checkpoints/regions/how-image-comparison-works.html]
Pixel tolerence is a vlaue that seted to determine that if the two images to besimilar.
It allowed number of dissimilar pixels. 
If the number of different pixels is less than or equal to pixelTolerance, 
then it can be considered the images to be identical.
For example, there are two images that has 2 dissimilar pixels,
if the pixel tolerance is 2, then these two images are identical.

4)Using color tolerance:
This is a range of value of the rgb value(0-255) that needed to be set to determine if the picture is similar,
Two pixels are considered identical if the difference between intensities of each of their color components does not exceed the specified value.
For example,when colortolerance is 0, which is the default value, the compared pixels are considered identical only if they have exactly the same color. 
When colortolerance is 255, pixels of any color are considered identical.


\Second{Watermarks}
% ref. Frequency_Domain_Watermarking_An_Overview.
In general, 
watermarks are symbols which are placed into physical objects such as documents, photos, etc. and their purpose is to carry information about objects’ authenticity.
[I. J. Cox, M. L. Miller, J. A. Bloom, J. Fridrich and T. Kalker, “Digital Watermarking and Steganography,” 2nd Edition, Morgan Kaufmann, Burlington, 2008.   [Citation Time(s):2]]
Watermarks are the most common way to protect images.
There are many capable watermarks for example Fourier Watermarking and other feasible watermarks.
In the following, I will list the most popular and helpful watermarks nowadays(2 different algorithms)

a) Fourier Watermarking[Poljicak, A. et al. “Discrete Fourier transform-based watermarking method with an optimal implementation radius.” J. Electronic Imaging 20 (2011): 033008.]
Fourier watermarking are using a technology called fourier transform.
Watermarks are embedded in magnitude of the Fourier transformation.
Fourier transform is one of the multidimensional transforms which converts a signal from a time/space domain representation to a frequency domain representation.[Wikipedia]


b)Robust Digital Watermarking




\Third{Image clustering}
Image clustering is one of the technology that grouping images into clusters such that the images within the same clusters are similar to each other, \
while those in different clusters are dissimilar to each other.
This can also be used in the first step of the reverse image search,
because the database has so many different kinds of images,
it's efficient to first do a image clustering.
There are many clustering methods in the following:

1. color-based pixel
1920*1080, computation is huge
definition [paper] \cite{}
history, why this exists, 
how to implement == principle.
pros cons
applications

2. visual information retrieval
https://en.wikipedia.org/wiki/Information_retrieval
item classification.


3. todo survey, if this algo exists [cite this]
if not exist, write into Proposed Method.

4.k-means





thought:
1. target image: opencv get its sketch_t ���軭.  feature engineering.

hough line transformation

benefits:
applications:
computation, storage (feature vectors, [90,90,40,40,0], [90,90,10,10,10];
        machine learning model ).

2. get set_of_sketches of images from tineye
3. search if sketch_t is in set_of_sketches, set()

{
'ladder': [ index of images];
'railroad': [ index of images];
'road': [ index of images];
}
//cannot understand this part
















Sketch
# reverse image searching
    color-based

    feature_extraction   

# copyright protection
    Watermarks  nft fourier frequency domain
# Image clustering
    - color-based
    - information-based [google pic...a ]
    https://patents.google.com/patent/US7801893B2/en  
    https://www.researchgate.net/publication/228943913_Clustering_visually_similar_images_to_improve_image_search_engines  {

To cluster images into groups of visually similar images we propose to use the feature histograms as proposed above to represent the images and two well known clustering methods: k-means [McQueen 67] and LBG clustering [Dempster & Laird+77, Linde & Buzo+80]. 
    }
        - visual information retrieval
         unstructured data
        - feature_extraction   
        - facial 
        - Image Segmentation  

Image Segmentation Using K -means Clustering Algorithm and Subtractive Clustering Algorithm��{
https://www.sciencedirect.com/science/article/pii/S1877050915014143
Subtractive clustering is a method to find the optimal data point to define a cluster centroid based on the density ofsurrounding data points9. This method is the extension of Mountain method, proposed by Chiu10. Mountain method is

}




a