\section{Introduction}
%1) Problem statement:
Recently in China, there are a lot of controversies about the copyright of pictures. This leads to a concept called image copyright. [citation from paper, ]
This problem has been troubled a lot of photographers, firms and companies.Many of them went on a court to protect their works. But this weren't work so well. Here is a example recent years in China.
There was an incident about a company called ChinaVisual
On Apr. 11th, 2019, the China Visual Group(CVG) has been alleged to have published with its own watermark on their website, this made the cyberspace affairs to suspend its website.
But the problem is,
the copyright of this picture belongs to an organization called Event Horizon Telescope. This organization has first released the picture to the whole world and made a claim that was not meant for any commercial use.
More over, people found other images of national flags which definitely not belongs to ChinaVisual.
This kind of incidents don't just happen in China, things like stealing images and using illegally happens all over the world.

The objective of me is to conduct a survey regarding to solving the issue of protection of image copyright.

\section{Related Works}
\begin{comment}


\end{comment}

This paper is going to discover this issue into two sub-topics: images clustering(comparison)and watermarking.

Q: How can we know when our image is stole by other people?
Their are two ways we can find know that our images are stolen by others.

1)reverse image searching
The definition of the reverse image searching is a search engine technology that takes an image file as input query and returns results related to the image.
You can get the information containing:
-similar images
-the list of websites that contain the images
-other dimensions of pictures you searched with

The reverse image searching works when you have an image to input with.
For example, if you have an image of cat,
you can put it into the reverse image searching,
and it will provide you thousands of similar images and also the information about the image whatever posted online.

There are many mature reverse searching engine consisting google, tineye, noobox and so on.
Many companies and individuals are using these engine in their daily life.

Q:How does the reverse image searching works?(image comparison)

Most of the principle of searching engines are color-based
The method is simply find the red, green, blue value(RGB value) for each images and do comparison.
RGB value is an additive color model in which red, green, and blue light are added together in various ways to reproduce a broad array of colors.
This method first collects the RGB value in each of the images and do comparison within them. % more related surveys

There are also other methods that can be used:
-Using Pixel tolerance:
%information
-Using color tolerance:
%information



2)Watermarks  % ref. Frequency_Domain_Watermarking_An_Overview.
todo: definition watermark. resource: wiki, paper. pros & cons
I collected the information that the most popular method is called Fourier Watermarking and other feasible watermarks.
a) Fourier Watermarking
todo: definition fourier watermark. resource: wiki, paper. pros & cons, image [paper name]
b)






# Image clustering
1. color-based pixel, 1920*1080, computation is huge
definition [paper] \cite{}
history, why this exists, 
how to implement == principle.
pros cons
applications

2. visual information retrieval
https://en.wikipedia.org/wiki/Information_retrieval
item classification.


3. todo survey, if this algo exists [cite this]
if not exist, write into Proposed Method.

thought:
1. target image: opencv get its sketch_t ���軭.  feature engineering.

hough line transformation

benefits:
applications:
computation, storage (feature vectors, [90,90,40,40,0], [90,90,10,10,10];
        machine learning model ).

2. get set_of_sketches of images from tineye
3. search if sketch_t is in set_of_sketches, set()

{
'ladder': [ index of images];
'railroad': [ index of images];
'road': [ index of images];
}

This paper is going to discover this issue into two sub-topic: image cluster and water mark.

Sketch
# reverse image searching
    color-based

    feature_extraction   

# copyright protection
    Watermarks  
# Image clustering
    - color-based
    - information-based [google pic...a ]
    https://patents.google.com/patent/US7801893B2/en  
    https://www.researchgate.net/publication/228943913_Clustering_visually_similar_images_to_improve_image_search_engines  {

To cluster images into groups of visually similar images we propose to use the feature histograms as proposed above to represent the images and two well known clustering methods: k-means [McQueen 67] and LBG clustering [Dempster & Laird+77, Linde & Buzo+80]. 
    }
        - visual information retrieval
         unstructured data
        - feature_extraction   
        - facial 

        - Image Segmentation  

Image Segmentation Using K -means Clustering Algorithm and Subtractive Clustering Algorithm��{
https://www.sciencedirect.com/science/article/pii/S1877050915014143
Subtractive clustering is a method to find the optimal data point to define a cluster centroid based on the density ofsurrounding data points9. This method is the extension of Mountain method, proposed by Chiu10. Mountain method is

}

k-means: 
a mixture of Gaussians: 
