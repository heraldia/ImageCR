\section{Introduction}
Problem statement:
Recently in China, 
there are a great deal of controversies about using pictures of others that is unauthorised.
This leads to a concept called image copyright.
Image copyright is used to protect the images owned by poeple or companies from copy or reproduce by others automatically. % ref. https://www.pixsy.com/academy/image-user/using-copyrighted-images/
However, a large quantity of firms and individuals didn't obey the rules.
This stealing problem has been troubled a large amount of photographers, firms and companies.
Many of them went on a court inorder protect their works by means of law.
But this weren't work so well because althogh there are copyright, 
there are still poeple using illegal images  without permisiion.
According to the information that was published on the website poeple.com.cn,
From<China Online Copyright Protection Annual Report 2018>, 
it pointed out that the report pointed out that copyright infringement cases of picture works have exploded, 
plagiarism and illegal reprinting on public accounts are serious.
In 2018, China made outstanding achievements in the protection of online copyright. 
During the "Swords Net 2018" special action, copyright law supervision departments investigated and dealt with 544 online infringement and piracy cases, 
including 74 criminal cases with a total value of 150 million RMB.

Here is a example recent years in China.
There was a company called ChinaVisual.
On Apr.11th,2019, 
the China Visual Group(CVG) has been alleged to have published with its own watermark on their website, 
this made the cyberspace affairs to suspend its website.
But the problem is, the copyright of this picture belongs to an organization called Event Horizon Telescope. 
This organization has first released the picture to the whole world and made a claim that was not meant for any commercial use.
More over, people found other images of national flags which definitely not belongs to ChinaVisual.
There are also discussions on a chinese social media called weibo about the image copyright of this incident. 
The Central Committee of the Communist Young League warned the ChinaVisual by using rhetorical question--Are those images of national flags yours?
This kind of incidents didn't just happen in China, things like stealing and using images illegally happens all over the world.

Nowadays, people are putting more importance on the image copyright problems and this lead us to think,
are there better and more effective way to protect the images?
The objective of this paper is to conduct a survey regarding to solve the issue of protection of image copyright and explain how they are using the technologies.

\section{Related Works}
There have been many things can be done in the protection of the image copyright. 
I seperate this section into three catagories: 
-Reverse image searching
-Watermarks-(fourier, basic RBG, frequency domain)
-Image clustering and so on.

\First{Reverse image searching}
There are many cases that poeple don't know their image has been used by other illegally.
Therefore, people need tools to find out how can they know that their works has been stolen.

The best mothod is reverse image searching.
Reverse image searching is a searching engine technology that takes an image file as input query and returns results related to the image.% ref. https://whatis.techtarget.com/definition/reverse-image-search][Gaillard, Mathieu, Elöd Egyed-Zsigmond, and Michael Granitzer. "CNN features for Reverse Image Search." Document numerique 21.1 (2018): 63-90.
Using reverse image search can get these imformations: % ref. https://tineye.com/faq#why
-Find sources of images
-Identify duplicate images(illegally stolen)
-Find higher resolution of images
-Locate web pages of the images

To better understand reverse image search, there are two concepts:
1) Image Retrieval
Image retrival is a system that normally used in the browsing.
By searching and retrieving images from the large database of images.
Examples like Baidu Image and Google image, users input text queries and will have a list of images which matches the text queries in return.
Those images are assosated with text such as keywords, tags and so on.
2) Content-based Image Retrieval(CBIR)
Reverse image search(RIS) is include in this system.
The aim of this system is to find a image that similar to the input images.
Examples for this system are Google images or Baidu Image and so on,
if someone submits a picture of a cat, in return he can get a list of similar images that similar to the input image but nor exacly the same.
Other mature reverse searching engine consisting Tineye, Noobox and so on.

Most of the IR system contains the following two steps:
1.(indexing)  for each image in a database, 
feature vectors capturing certain essential properties of the image is conserved in a feature- base
2.(searching)given a query image, its feature is computed and compared to other feature vectors of images.
Last, the similar images will be returned to users.

There are a plenty of good algorithms to attain the reverse image searching.
The base of the reverse image seaching is do comparsion among images.

1) Color histogram and Color correlogram % ref. Image Indexing Using Color Correlograms
Normallly, color histograms are used as feature vectors of different images,
it describes the global color distribution, it's representation of the distribution of colors in an image. %rdf. Wiki.
These image representations in this circumstances is an interger vector called N-bins.
Although it's a relativly sensitive to small changes of the images.
It has no spatial information and therfore leads to incorrect resorts.

Color correlogram is a new image feature and it is used for image indexing and comparsion.
This feature extracts the spatial correlation of colors, 
and is both effective and inexpensive for content-based image retrieval.

In comparison with the histogram methods, color histograms can operate better in this four aspects:
1. it contains spatial correlation of colors.
2. it is easy to compute.
3. it can be used to describe the global distribution of local spatial correlation of colors.
4. the size of feature of the images is small.

The color coerrelogram is a method that expresses how the spatial correlation of pairs of colors changes with distance besides the color distribution in an image.
First,the autocorrelogram of the image will capture the spatial correlation between identical colors.
Next, the correlogram is needed to be computed.
A team in Cornell University uses quantities. 
They defined quantities which calculate the number of pixels of a given color within a given distance from a fixed pixel in two directions: positive horizontal and positive vertical.
Last, we need to compare. 
These problems are normally resolved by measuring the distance through the subtraction of the feature vectors.


2)Convolutional Neural Networks(CNN) %ref. CNN features for Reverse Image Search
CNN is a concept in deep learning, it mostly applied to analyzing visual imagery in computer vision.
CNNs are regularized versions of multilayer perceptrons. 
Multilayer perceptrons usually mean fully connected networks, that is, each neuron in one layer is connected to all neurons in the next layer. % ref. Wiki
With these connective layers, it can take advantages of natural signals, 
these natural signals are often locally highly correlated.
And also their local statistics of images are invariant to location. 
If a pattern can appear in one location of an image, it could also appear somewhere else in this image.
CNN contains three different layers:
1. The convolution layers, the pooling layer and the Relu
a) The convolution layer applies a set of discrete convolutions on the output of the previous layer. 
Each convolution is done using a kernel and produces a 2D feature map. 
The feature maps contain high activations if their convolutions have detected an interesting motif. 
This layer can be seen as a layer with connections, 
so that a neuron can be connected to neurons of the previous layer in the same region. 
b) The ReLU applies a non-linear function to the output of the convolution layer. 
Other non-linear activation functions can be applied: 1: hyperbolic tangent 2:sigmoid function 
c) The pooling layer seperates the output of the previous layer into a grid, 
then on each cell of the grid a pooling function is applied and outputs only a single value.
This layer intergrates the semantically similar features into one.
Last, this function will output a maximum value of a local patch.
2. the fully connected layers and the loss layer



By means of the CNN, the class of the network is able to automatically extracted the features from the dataset
CNN is efficient and widely used because of the amount of data and the computaion power.




3) Perceptual Hashing funtion






4)

\Second{Watermarks}
% ref. Frequency_Domain_Watermarking_An_Overview.
In general, 
watermarks are symbols which are placed into physical objects such as documents, photos, etc. and their purpose is to carry information about objects’ authenticity.
% I. J. Cox, M. L. Miller, J. A. Bloom, J. Fridrich and T. Kalker, “Digital Watermarking and Steganography,” 2nd Edition, Morgan Kaufmann, Burlington, 2008.   [Citation Time(s):2]
Watermarks are the most common way to protect images.
There are many capable watermarks for example Fourier Watermarking and other feasible watermarks.
In the following part, I will list the most popular and helpful watermarks nowadays(2 different algorithms)

a) Fourier Watermarking[Poljicak, A. et al. “Discrete Fourier transform-based watermarking method with an optimal implementation radius.” J. Electronic Imaging 20 (2011): 033008.]
Fourier watermarking are using a technology called fourier transform.
Watermarks are embedded in magnitude of the Fourier transformation.
Fourier transform is one of the multidimensional transforms which converts a signal from a time/space domain representation to a frequency domain representation.[Wikipedia]


b)Robust Digital Watermarking




\Third{Image clustering}
Image clustering is one of the technology that grouping images into clusters such that the images within the same clusters are similar to each other, \
while those in different clusters are dissimilar to each other.
This can also be used in the first step of the reverse image search,
because the database has so many different kinds of images,
it's efficient to first do a image clustering.
There are many clustering methods in the following:

1. color-based pixel
1920*1080, computation is huge
definition [paper] \cite{}
history, why this exists, 
how to implement == principle.
pros cons
applications

2. visual information retrieval
https://en.wikipedia.org/wiki/Information_retrieval
item classification.


3. todo survey, if this algo exists [cite this]
if not exist, write into Proposed Method.

4.k-means










Sketch
# reverse image searching
    color-based

    feature_extraction   

# copyright protection
    Watermarks  nft fourier frequency domain
# Image clustering
    - color-based
    - information-based [google pic...a ]
    https://patents.google.com/patent/US7801893B2/en  
    https://www.researchgate.net/publication/228943913_Clustering_visually_similar_images_to_improve_image_search_engines  {

To cluster images into groups of visually similar images we propose to use the feature histograms as proposed above to represent the images and two well known clustering methods: k-means [McQueen 67] and LBG clustering [Dempster & Laird+77, Linde & Buzo+80]. 
    }
        - visual information retrieval
         unstructured data
        - feature_extraction   
        - facial 
        - Image Segmentation  

Image Segmentation Using K -means Clustering Algorithm and Subtractive Clustering Algorithm��{
https://www.sciencedirect.com/science/article/pii/S1877050915014143
Subtractive clustering is a method to find the optimal data point to define a cluster centroid based on the density ofsurrounding data points9. This method is the extension of Mountain method, proposed by Chiu10. Mountain method is

}




academy